\documentclass[12pt]{article}
%\usepackage{times}
\usepackage{cite}
%this is a comment
\title{Trash Drone}
\author{Pavel Shonka, shonkap}




\begin{document}
\maketitle
\tableofcontents



\section{Trash Drone}
Uber is a global tech company that has revolutionized transportation using solely web and mobile software applications. Name is the next Uber that will revolutionize trash collection using web and mobile software applications. The current goal of the project is using drone enthusiasts drones to analyze litter on beaches and picking it up to keep our beaches clean. By using Amazon Recognition, the drones will be able to upload pictures taken by a quality camera and analyze all objects that are considered trash and need to be picked up. There are drones at the moment that can do what this project is proposing but it currently can not pick up trash using the drones. In short, the goal is to reduce trash in beaches using volunteering drones that are programmed by Name's software that can be uploaded directly to the flight controller.
\newline
\newline

\subsection{Main Problems}
This problem has been all over the news, the Senior Vice President expresses "Our ocean is sick" \cite{pfleeger2010software}. Odds are you see it on the news daily where some beach has been closed off due to pollution or some animal has died after digesting plastics. If you have ever been to a beach you have seen some form of litter on it. This is a growing issue as the human population increases and more people visit the beaches where more trash finds its way to the ocean and eventually on to the nice pristine beaches.
\newline
\newline

\subsubsection{The Reality}
Finding trash on the beach can be anywhere from just being grossed out or not wanting to stay there all the way to being injured by broken glass, nails or even worse a needle. Although not many studies have been done on the percentage of people who receive injuries on the beach the few that have found it to be about 21.6 percent of people injured at the beach are due to beach litter. That is relatively high for something that shouldn't even be on the beach in the first place.
\newline
\newline

\subsubsection{Approach}
With the amount of trash that we currently see in the world, I believe the primary trash solution might be a software application since software can be distributed worldwide really quick to drones. The first step would be having a drone that has a compatible camera that will work with the software as well as grabbing mechanisms that can be added onto drones universally. One piece of information that has not been mentioned so far is that the drone will be able to autonomously take off and scan the floor using a recognition software. If there is multiple drones, the software will allow them to communicate so they can efficiently clean our beaches for us. 
\newline
\newline

\subsubsection{Why ours is better}
The difference between our approach and others approaches are that we hope to allow other drones like Uber allows other cars to make their product accessible as to reduce costs and allow for everybody to help in the process of collecting litter from our beaches. We are also hoping to make drones that are directly engineered to pick up trash which would hopefully bring the price down as well as make them more effective. The app can also allow a swarm like function if there are enough drones telling each one to go to a different section of the beach to pick up trash speeding up the process unlike current designs where it is using one or two drones only. With people using their own drones the maintenance fee will be substantially small as drones are switched out frequently and each owner will maintain their own.
\newline
\newline

\subsubsection{What we will need}
Currently the only resources needed will be a server to run the mobile app in order to figure out which drones are going to which litter as well as where the litter is located. A 2.4GHz transmitter as well as any other transmitters that are needed to talk to the drones will also be needed. Everything but where the location of the trash is and where each vehicle is headed to will be run by each vehicle autonomously allowing us to only need a small database for location storage.
\newline
\newline

\subsubsection{Problems}
Unfortunately as an after thought using flying drones would be highly ineffective so we decide that it should be done by autonomous vehicles on the ground as they can carry a lot larger payload, travel much further, and would be cheaper than using the flying counter part. By using a vehicle the maintenance should be very small and a lot more reliable. There is the risk of someone stealing one or it having an issue where it gets destroyed by a storm or taken into the ocean by a wave but hopefully the GPS and external sensors should help to mitigate the risk.



\bibliography{myref}
\bibliographystyle{plain}

\end{document}
